\documentclass{article}
\usepackage{cite}
\usepackage[utf8]{inputenc}
\usepackage{hyperref}
\usepackage[hypcap]{caption}
\usepackage[english]{babel}
\usepackage{url}
\usepackage{amsmath}
\usepackage{pgfplots}
\usepackage{color}
\usepackage[margin=1in]{geometry}
\setlength\belowcaptionskip{-0.5cm}
\captionsetup[figure]{font=small,labelfont=small}


\title{Derivation of $\epsilon$}
\author{Ilaria Manco \& Jim Bremner}
\date{15/01/18} % delete this line to display the current date

%%% BEGIN DOCUMENT
\begin{document}

\maketitle

\section{Error Function}
We derive the analytical form of the error $\epsilon$ 
\begin{equation}\label{eq:1}
\epsilon =  \frac{T_{ib} - (T_{ij} + T_{ik})}{T_{ib}}
\end{equation}
for the case of the gravity model with an exponential deterrence function:
\begin{equation}\label{eq:2}
T_{ij} = k_{ij} m_j f(r_{ij}),
\end{equation}
where 
\begin{equation}\label{eq:3}
k_{ij}^{-1} = \sum_j{m_j f(r_{ij})}
\end{equation}
and
\begin{equation}\label{eq:4}
f(r_{ij}) = e^{-\gamma r_{ij}}
\end{equation}

Substituting \eqref{eq:2} into \eqref{eq:1} gives
\begin{equation}\label{eq:5}
\epsilon = 1 - \left( \frac{k_{ij} m_j}{e^{\gamma r_{ij}}} + \frac{k_{ik} m_k }{e^{\gamma r_{ik}}} \right) \frac{e^{\gamma r_{ib}}}{k_{ib} m_b}
\end{equation}

{\color{red} We obtain a continuous form of $k_{ib}$ by approximating the sum to an integral. We then use a polar coordinates system with location $k$ on the x-axis and perform an integral over the circular sector between the x-axis and $2\pi - \delta \theta$. Assuming $\delta \theta$ is small enough, this can approximate the area of interest sufficiently well.

\begin{equation}\label{eq:6}
\begin{aligned}
k_{ib}^{-1} &= \int_{A}{e^{-\gamma r_{ib}} dm} \\
& = \int_{\theta = 0}^{2\pi - \delta \theta} \int_{r_min}^{r_{max}}{\rho e^{-\gamma r_{ib}} r dr d\theta} \\
& = \rho \frac{2\pi - \delta \theta}{\gamma^2} \left[e^{-\gamma r_{min}} \left( \gamma r_{min} +1 \right) - e^{-\gamma r_{max}} \left( \gamma r_{max} +1 \right) \right]
\end{aligned}
\end{equation}
[Note: we realised we might not need this form of $k$, since we can find an expression for $\epsilon$ by keeping $k$ and decomposing the sum in the following way:]}

\begin{subequations}
\begin{equation}\label{eq:7a}
k_{ib}^{-1} = \sum_{l \neq b}{m_i e^{-\gamma r_{il}}} + m_b e^{-\gamma r_{ib}}
\end{equation}
\begin{equation}\label{eq:7b}
k_{ij}^{-1} = \sum_{l \neq j, k}{m_i e^{-\gamma r_{il}}} + m_j e^{-\gamma r_{ij}} + m_k e^{-\gamma r_{ik}}
\end{equation}
\end{subequations}

Since $k_{ij} = k_{ik}$, substituting \eqref{eq:7a} and \eqref{eq:7b} into \eqref{eq:5} gives

\begin{equation}\label{eq:8}
\epsilon = 1 - \frac{ m_j e^{-\gamma r_{ij}} + m_k e^{-\gamma r_{ik}}}{m_b e^{-\gamma r_{ib}}} 
\frac{k_{ib}}{k_{ij}}
\end{equation}

Provided that $k_{ib} \simeq k_{ij}$, the expression for $\epsilon$ can be reduced to
\begin{equation}\label{eq:9}
\epsilon = 1 - \frac{m_j e^{-\gamma r_{ij}} + m_k e^{-\gamma r_{ik}}}{m_b e^{- \gamma r_{ib}}}
\end{equation}

From \eqref{eq:7a} and \eqref{eq:7a}, we see that the assumption  $k_{ib} \simeq k_{ij}$ is valid as long as 
\begin{subequations}
\begin{equation}\label{eq:10a}
m_b e^{- \gamma r_{ib}} \ll \sum_{l \neq b}{m_i e^{-\gamma r_{il}}}
\end{equation}
and 
\begin{equation}\label{eq:10b} 
{m_j e^{-\gamma r_{ij}} + m_k e^{-\gamma r_{ik}}} \ll \sum_{l \neq j, k}{m_i e^{-\gamma r_{il}}}.
\end{equation}
\end{subequations}

Provided that $N = \sum_{j}{1}$ is sufficiently large, the expressions above hold and we can then assume \eqref{eq:9} is a valid approximation.
\\
\par Finally, noting that
\begin{equation}\label{eq:11}
r_{ij} \simeq r_{ik} \simeq \sqrt{r_{ib}^2 + \left(\frac{r_{jk}}{2}\right)^2}
\end{equation}

\eqref{eq:9} can be rewritten as
\begin{equation}\label{eq:12}
\epsilon \left( r_{ib}, r_{jk} \right)_{e} = 1 - e^{-\gamma \left( r_{ij} - r_{ib}\right)} 
\end{equation}
\\
The same derivation can be followed for the power law form of the deterrence function $f(r_{ij}) = r_{ij}^{-\gamma}$ to obtain 

\begin{equation}\label{eq:13}
\epsilon(r_{ib}, r_{jk})_{p} = 1 - \frac{m_j r_{ij}^{-\gamma} + m_k r_{ik}^{-\gamma}}{m_b r_{ib}^{-\gamma}}
\end{equation}
\end{document}