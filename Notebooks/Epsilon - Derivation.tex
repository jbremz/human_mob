\documentclass{article}
\usepackage{cite}
\usepackage[utf8]{inputenc}
\usepackage{hyperref}
\usepackage[hypcap]{caption}
\usepackage[english]{babel}
\usepackage{url}
\usepackage{amsmath}
\usepackage{pgfplots}
\usepackage[margin=1in]{geometry}
\setlength\belowcaptionskip{-0.5cm}
\captionsetup[figure]{font=small,labelfont=small}


\title{Derivation of $\epsilon$}
\author{Ilaria Manco \& Jim Bremner}
\date{15/01/18} % delete this line to display the current date

%%% BEGIN DOCUMENT
\begin{document}

\maketitle

\section{Error Function}
We derive the analytical form of the error $\epsilon$ 
\begin{equation}\label{eq:1}
\epsilon =  \frac{T_{ib} - (T_{ij} + T_{ik})}{T_{ib}}
\end{equation}
for the case of the gravity model with an exponential deterrence function:
\begin{equation}\label{eq:2}
T_{ij} = k_{ij} m_j f(r_{ij}),
\end{equation}
where 
\begin{equation}\label{eq:3}
k_{ij}^{-1} = \sum_j{m_j f(r_{ij})}
\end{equation}
and
\begin{equation}\label{eq:4}
f(r_{ij}) = e^{-\gamma r_{ij}}
\end{equation}

Substituting \eqref{eq:2} into \eqref{eq:1} gives
\begin{equation}\label{eq:5}
\epsilon = 1 - \left( \frac{k_{ij} m_j}{e^{\gamma r_{ij}}} + \frac{k_{ik} m_k }{e^{\gamma r_{ik}}} \right) \frac{e^{\gamma r_{ib}}}{k_{ib} m_b}
\end{equation}

We obtain a continuous form of $k_{ib}$ by approximating the sum to an integral. We then use a polar coordinates system with location $k$ on the x-axis and perform an integral over the circular sector between the x-axis and $2\pi - \delta \theta$. Assuming $\delta \theta$ is small enough, this can approximate the area of interest sufficiently well.

\begin{equation}\label{eq:6}
\begin{aligned}
k_{ib}^{-1} &= \int_{A}{e^{-\gamma r_{ib}} dm} \\
& = \int_{\theta = 0}^{2\pi - \delta \theta} \int_{r_min}^{r_{max}}{\rho e^{-\gamma r_{ib}} r dr d\theta} \\
& = \rho \frac{2\pi - \delta \theta}{\gamma^2} \left[e^{-\gamma r_{min}} \left( \gamma r_{min} +1 \right) - e^{-\gamma r_{max}} \left( \gamma r_{max} +1 \right) \right]
\end{aligned}
\end{equation}

Since $k_{ij} = k_{ik}$, substituting \eqref{eq:6} into \eqref{eq:5} gives

\begin{equation}\label{eq:7}
\epsilon = 1 - \frac{ m_j e^{-\gamma r_{ij}} + m_k e^{-\gamma r_{ik}}}{m_b e^{-\gamma r_{ib}}} 
\frac{ \left[e^{-\gamma r_{min}} \left( \gamma r_{min} +1 \right) - e^{-\gamma r_{max}} \left( \gamma r_{max} +1 \right) \right] + m_b e^{-\gamma r_{ib}}} { \left[e^{-\gamma r_{min}} \left( \gamma r_{min} +1 \right) - e^{-\gamma r_{max}} \left( \gamma r_{max} +1 \right) \right] + m_j e^{-\gamma r_{ij}} + m_k e^{-\gamma r_{ik}}}  
\end{equation}

Provided that $r_{ib} \gg r_{jk}$, then we can approximate $r_{ij} \simeq r_{ik} \simeq r_{ib}$. Given that $ m_b = m_i + m_j$, we can then assume 
\begin{equation}\label{eq:8}
m_b e^{-\gamma r_{ib}} \simeq m_j e^{-\gamma r_{ij}} + m_k e^{-\gamma r_{ik}}
\end{equation}

The expression for $\epsilon$ then simplifies to
\begin{equation}\label{eq:9}
\epsilon \left( r_{ib}, r_{jk} \right)_{e} = 1 - \frac{m_j e^{-\gamma r_{ij}} + m_k e^{-\gamma r_{ik}}}{m_b e^{- \gamma r_{ib}}}
\end{equation}

Noting that
\begin{equation}\label{eq:10}
r_{ij} \simeq r_{ik} \simeq \sqrt{r_{ib}^2 + \left(\frac{r_{jk}}{2}\right)^2}
\end{equation}

\eqref{eq:9} can be rewritten as
\begin{equation}\label{eq:11}
\epsilon \left( r_{ib}, r_{jk} \right)_{e} = 1 - e^{-\gamma \left( r_{ij} - r_{ib}\right)} 
\end{equation}


\begin{tikzpicture}
\draw [-latex, thick] (0, 0) -- (0, 6) node  [above] {\Large{$y$}};
\draw [-latex, thick] (0, 0) -- (6, 0) node  [right] {\Large{$x$}};
\draw[dotted] (0,0)--(4, 0);
\draw[dotted] (0,0)--(5,3) ;  
\node at (4, 0) {\textbullet \\ $k$};
\node at (5, 3) {\textbullet \\ $j$};
\node at (2.5, 0.5) {$\delta \theta$};
\draw (2, 0) arc[radius=3cm,start angle=0,end angle=21.3];
\end{tikzpicture}
\end{document}